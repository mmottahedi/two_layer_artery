\documentclass[review]{elsarticle}
\usepackage{lineno,hyperref}
\modulolinenumbers[5]
%\journal{journal of \LaTeX\ Templates}
%\bibliographystyle{elsarticle-num}
\usepackage{amsmath}
%\documentclass[11pt]{article}
%\usepackage{fullpage}

\begin{frontmatter}
\title{Two layer Artery Model with Distributed Fiber Orientation}
\author{Mohammad Mottahedi\fnref{myfootnote}}
\address{address}
\begin{document}
\maketitle
\begin{abstract}
this is abstract
\end{abstract}

\begin{keyword}
this is keyword
\end{keyword}

\end{frontmatter}


\section{Methodology}


\subsection{Experimental Measurements}
\paragraph{Preconditioning and biaxial test} Intact Artery. To determine the stress-strain relationship of porcine artery, the axial extension and radial
inflation were measured under internal lumen pressure. The artery was mounted on the cannula horizontally on one end and tied to a luer stopper on the
other end to allow for free axial extension. The cannula was connected to a pressure meter and syringe pump filled with PBS solution. The artery was 
preconditioned to obtain reproducible mechanical data by inflating the artery to a $300 mmHg$ pressure for five cycle. After preconditioning the
the artery was slowly inflated and the inflated outer diameter and extended length of the artery was measured from digital images taken during the inflation test.
\paragraph{Individual layers biaxial test} After completing the biaxial test on the intact artery, the adventitia layer was carefully dissected 
without damaging the Media and Intima layers. Intima-Media layer was tested using the biaxial test protocol described above. 
\paragraph{No-load and Zero Stress State} To obtain the no-load opening angle, rings were cut off from proximal and distal ends of the artery and
photographed under zero load condition. The rings were cut off radially and left in PBS solution for a while to relax and photographed again for 
morphological analysis.

\subsection{Model equations}

The arteries are modeled as two-layers with separate mechanical properties for the intimia-media and adventitia layers. The arterial layer was modeled
as fiber-reinforce composite in which two families of collagen fibers are embeded in ground matrix by means of an incompressible isotropic \emph{neo-Hookean} model.  the arteries are considered with 
an opening angle $\Phi_0$ to account for residual stress in cirmuferencial direction. At zero stress state the inner radius, outer radius, and length of the artery 
are defined by $R_{i}^A,R_{e}^A,L$ for adventitia layer and
$R_{i}^IM,R_{e}^{IM},L$ for intimia-media layer. Similarly the inner radius, outer radius, and
length at deformed state under internal pressure $p$ are $r_{i}^A,r_{e}^A,l$
and $r_{i}^{IM},r_{e }^{IM},l$.

In cylindrical coordinates the deformation gradient matrix $F$ is:

\begin{equation}
  F=  \left[\begin{array}{ccc}
        \frac{dr}{dR} & 0 & 0 \\
        0 & \frac{2\pi}{\Theta_0}\frac{r}{R} & 0 \\
    0 & 0 & \lambda_z \end{array} \right]
    \end{equation}

The Green strain tensor is
\begin{equation}
    E=\frac{1}{2}(F^TF-I)
\end{equation}

with non-zero components
\begin{equation}
    E_r=\frac{1}{2}(\lambda_r^2-1), E_{\theta}=\frac{1}{2}(\lambda_{\theta}^2-1), E_z=\frac{1}{2}(\lambda_z^2-1) 
\end{equation}

where
\begin{equation}
    \lambda_r=\frac{\partial r}{\partial R}, \lambda_{\theta}=\frac{2\pi}{\Theta_0}\frac{r}{R}, \lambda_z=\lambda_z
\end{equation}

where $\Theta_0$ is equal to $2\pi-\Phi_0$. Enforcing incompressibility $(detF=1)$

\begin{equation}
    \lambda_r\lambda_{\theta}\lambda_z=1, \frac{\partial r}{\partial R}=\frac{\Theta_0 R}{\pi r \lambda_z}
\end{equation}

The strain energy fuction for each layer is given by %\cite{key1}
\begin{multline} 
    \Psi_{IM}=\frac{C_{IM}}{2}(I_1-3)+\Sigma_{i=1,2} \frac{k_{1i,IM}}{2k_{2i,IM}}[e^{k_{2i,IM}[\kappa_{i,IM}I_1+(1-3\kappa_{i,IM})I_4-1]^2}-1] \\
     R_i \leq R \leq R_i+H_m 
\end{multline}
\begin{multline}
    \Psi_{A}=\frac{C_A}{2}(I_1-3)+\Sigma_{i=1,2} \frac{k_{1i,A}}{2k_{2i,A}}[e^{k_{2i,A}[\kappa_{i,A}I_1+(1-3\kappa_{i,A})I_4-1]^2}-1]\\
     R_i+H_m \leq R \leq R_o
\end{multline}

where $k_{1i},k_{2i}$ are material constants. $I_1=tr(F^TF)$ is the first invariant and
$I_4=\lambda_z^2\sin^2(\alpha_0)+\lambda_{\theta}^2 \cos^2(\alpha_0)$ is a measure of stretch in the $i$th family of fiber. $\alpha_0$ is the
angle between the fiber and radial direction of the artery. $\kappa$ is the disperssion parameter between $[0,\frac{1}{3}]$ and defined by
\begin{equation}
    \kappa=\frac{1}{4}\int_0^{\pi} \rho(\theta) \sin^3(\theta)\theta d\theta
\end{equation}

$\rho(\theta)$ is the modified $\pi$-periodic \emph{von-Mises} distribution that was used to determine the collagen fiber distribution 
embeded in each layer.
\begin{equation}
    \rho(\theta)=4\sqrt{\frac{b}{2\pi}}\frac{e^{[b(\cos(2\theta)+1)]}}{erfi(\sqrt{2b})}
\end{equation}

Assuming that both fiber families have same angle with circumfrencial direction and material properties due to axi-symmetry of the arteries, i.e. $k_{11}=k_{12}$,$k_{21}=k_{22},\alpha_{01}=\alpha_{02}$
equations Eq.(6) and Eq.(7) become

\begin{multline}
    \Psi_{j}=\frac{C_j}{2}(2E_{r}+2E_{\theta}+2E_z)+\frac{k_{1,j}}{k_{2,j}}[e^{k_{2,j}[\kappa_{j}(2E_{r}+2E_{\theta}+2E_z)+(1-3\kappa_{j})[(2E_z+1)\sin^2(\alpha_0)+(2E_{\theta}+1)\cos^2(\alpha_0)]-1]^2}-1], \\
    j=IM,A
\end{multline}

the Cauchy stresses in cylinderical coordinates are given by
\begin{equation}
    \sigma_i=\lambda_i^2 \frac{\partial \Psi_j}{\partial E_i}+ K, \quad
    i=r,\theta,z, \quad j=IM,A
\end{equation}

K is the Lagrangian multiplier that enforces incompressibility.

The euqaiton of equilibrium in a straght axisymmetric artery is
\begin{equation}
    \frac{\partial \sigma_r}{\partial r}+\frac{\sigma_r-\sigma_\theta}{r}=0
\end{equation}
with boundary conditions
\begin{equation}
    \sigma(r_i)\mid_{r=r_i^{IM}}=-P;\sigma_r\mid_{r=r_o^{IM}}=\sigma_r\mid_{r=r_i^A};\sigma_r\mid_{r=r_o^A}=0
\end{equation}

Integration along boundary conditions yields
\begin{equation}
    P=\int_{r_{i}^{IM}}^{r_{e}^{IM}}(\lambda_\theta^2 \frac{\partial
    \Psi_{IM}}{\partial E_\theta} - \lambda_r^2 \frac{\partial
    \Psi_{IM}}{\partial E_r}) \frac{\partial r}{r}\\
    +\int_{r_{i}^A}^{r_{e}^A}(\lambda_\theta^2 \frac{\partial \Psi_{A}}{\partial E_\theta} - \lambda_r^2 \frac{\partial \Psi_{A}}{\partial E_r}) \frac{\partial r}{r}
\end{equation}uu

The axial force $(N)$ can be determined by
\begin{equation}
    N=\pi \int_{r_{i}^{IM}}^{r_{e}^{IM}}(2\lambda_z^2 \frac{\partial \Psi_{IM}}{\partial E_z}- \lambda_\theta^2 \frac{\partial \Psi_{IM}}{\partial E_\theta}- \lambda_r^2 \frac{\partial \Psi_{IM}}{\partial E_r})rdr 
    +\pi \int_{r_{i}^A}^{r_{e}^A}(2\lambda_z^2 \frac{\partial \Psi_{A}}{\partial E_z}- \lambda_\theta^2 \frac{\partial \Psi_{A}}{\partial E_\theta}- \lambda_r^2 \frac{\partial \Psi_{A}}{\partial Er})rdr                           
\end{equation}

\subsection{Determination of model parameters}
\subsubsection{Geometry of arterial ring, intimia-media and andventitia layer
thikness}


\paragraph{Determining Material Constants of individual layers} To determine the material constats $c,k1,k2$  the algorithm should minimize the square of the difference between experimental and theoretical values of 
Pressure $P_i$ and axial for $N$.
\begin{equation}
Error = \Sigma[(P_t -P_{exp})_n^2+(N_t-N_{exp})_n^2 ]
\end{equation}

%\section*{References}

\end{document}

